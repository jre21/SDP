\documentclass[12pt]{amsart}

\usepackage{fullpage,hyperref,url}
\usepackage[pdftex]{graphicx}

\title{Project Proposal: Quintic Spectrahedra}
\author{Jacob Emmert-Aronson}
\author{Moor Xu}
\date{March 12, 2014}
\begin{document} 
\maketitle

\emph{Spectrahedra} of degree $n$ in $\mathbb{R}^k$ are convex bodies
given by $k$-dimensional affine slices of the cone of $n \times n$
positive semidefinite matrices.  They arise as feasible domains in
semidefinite programming and each is described by a linear matrix
inequality.
\vspace\baselineskip

\begin{center}

\includegraphics[scale=.15]{pillow.jpg}
\vspace\baselineskip

{\small
\emph{The pillow}: the spectrahedron
$
\begin{pmatrix}
  1&x&0&x\\
  x&1&y&0\\
  0&y&1&z\\
  x&0&z&1
\end{pmatrix}
\succeq 0\text.$  Here $k=3,\, n=4$.}
\end{center}
\vspace\baselineskip

Because the cost function of an SDP is linear, the optimal point
always lies on the surface.  One interesting and practically useful
question is the likelihood for the result of an optimization to be a
node, one of the corner points seen when visualizing the
spectrahedron.  A generic matrix represented by a point on the surface
of the spectrahedron has rank $n-1$, while the matrix at a node
typically has rank $n-2$.  This low-rank property of nodes often
translates to easier computation.

We propose to study the nodal structure of quintic spectrahedra in
$\mathbb{R}^3$.  We wish to study the possible numbers of nodes on the
surface of a spectrahedron as well as real nodes on the symmetroid,
its Zariski closure.  We will also evaluate the effect of these
quantities on the probability that a random optimization problem will
select a node.

To carry this out, we will generate random spectrahedra and compute
the positions of nodes, while also running multiple optimization
problems on each.  Jacob Emmert-Aronson and Joe Kileel wrote much of
the code to carry this out last semester.  Currently, Singular is used
to determine locations of nodes through a Gr\"obner basis algorithm;
due to apparent numeric instabilities, however, this misidentifies
nodes in certain edge cases.  We hope to replace this with the
homotopy algorithms implemented in Bertini, which produce more
reliable results.  We will then check the most interesting edge cases
from the previous data set, to determine whether unusual behavior
persists under the more accurate analysis, and generate a new data set
to determine possible node counts with relative frequencies, and other
properties which may be of interest.

We also plan to learn what has been done in the case of quartic spectrahedra,
and see if any of that theory generalizes to the quintic case.


%%%%%%%%%%%%%%%%%%%%%%%
\bibliographystyle{amsplain}
\bibliography{references}
\nocite{*}

\end{document} 


